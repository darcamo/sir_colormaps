\documentclass{standalone}

\usepackage[utf8]{inputenc} % Use this if the file is encoded with utf-8
\usepackage{lmodern}  % Very good to use with the fontenc package to generate good PDFs
\usepackage[T1]{fontenc}  % Important. See http://tex.stackexchange.com/questions/664/why-should-i-use-usepackaget1fontenc
\usepackage{amsmath,amssymb} % Part of AMS-LaTeX
% One of the good things of the amsmath package is the math enviroments matrix, pmatrix, bmatrix, Bmatrix, vmatrix and Vmatrix
\usepackage{tikz}
\usetikzlibrary{positioning}
\usetikzlibrary{shadows}
\usetikzlibrary{backgrounds}
\usetikzlibrary{shapes}
\usetikzlibrary{shapes.multipart}
\usetikzlibrary{matrix}
\usetikzlibrary{intersections}
\usetikzlibrary{fit}
%\usetikzlibrary{decorations.pathmorphing}
\usetikzlibrary{decorations.pathreplacing}

\usepackage{pgfplots}
\pgfplotsset{compat=1.9}
\usepackage{pgfplotstable}

% My custom package with my math definitions
% Located at /home/darlan/Dropbox/Arquivos de Instalação/Latex_Packages/MathDefinitions.sty
\usepackage{MathDefinitions}

\begin{document}

\tikzset{AP/.style={draw, circle, fill=red, minimum size=8}}

\begin{tikzpicture}[scale=<<scale>>,transform shape]
  

  \begin{axis}[axis lines=none, 
    width=16cm,
    height=16cm]
    \addplot table[col sep=comma,x index=0, y index=1, only marks] {discretized_room.csv};
    \draw (axis cs: -1.0,-1.0) rectangle (axis cs:1,1);
    \draw (axis cs: 0,0) node[AP] {};
  \end{axis}
  
\end{tikzpicture}

\end{document}

%%% Local Variables:
%%% mode: tikz
%%% TeX-PDF-mode: t
%%% TeX-master: t
%%% End:
