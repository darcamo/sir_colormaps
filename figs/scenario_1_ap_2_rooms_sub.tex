\documentclass{standalone}

\usepackage[utf8]{inputenc} % Use this if the file is encoded with utf-8
\usepackage{lmodern}  % Very good to use with the fontenc package to generate good PDFs
\usepackage[T1]{fontenc}  % Important. See http://tex.stackexchange.com/questions/664/why-should-i-use-usepackaget1fontenc
\usepackage{amsmath,amssymb} % Part of AMS-LaTeX
\usepackage{siunitx}
% One of the good things of the amsmath package is the math enviroments matrix, pmatrix, bmatrix, Bmatrix, vmatrix and Vmatrix
\usepackage{tikz}
\usetikzlibrary{positioning}
\usetikzlibrary{shadows}
\usetikzlibrary{backgrounds}
\usetikzlibrary{shapes}
\usetikzlibrary{shapes.multipart}
\usetikzlibrary{matrix}
\usetikzlibrary{intersections}
\usetikzlibrary{fit}
%\usetikzlibrary{decorations.pathmorphing}
\usetikzlibrary{decorations.pathreplacing}
\usetikzlibrary{decorations.markings}
\usetikzlibrary{calc}

% My custom package with my math definitions
% Located at /home/darlan/Dropbox/Arquivos de Instalação/Latex_Packages/MathDefinitions.sty
\usepackage{MathDefinitions}

\begin{document}


\tikzset{AP/.style={draw, circle, fill=red, minimum size=10}}

\begin{tikzpicture}[scale=2]
  % We explicitly set the bounding box here so that the text nodes outside
  % the grid do not extend the bounding box.
  % \path[use as bounding box] (-6, -6) rectangle (6, 6);

  \draw[fill=gray] (-1.5, -2.5) grid (2.5, 2.5);

  % Draw the nodes
  \draw (-.5, 1.5) node[AP] (AP1) {1};
  \draw (1.5, 1.5) node[AP] (AP2) {2};
  \draw (0.5, 0.5) node[AP] (AP3) {3};
  \draw (-.5, -0.5) node[AP] (AP4) {4};
  \draw (1.5, -0.5) node[AP] (AP5) {5};
  \draw (0.5, -1.5) node[AP] (AP6) {6};

  % Draw the interesting points
  \draw[fill=green,xshift=-1pt,yshift=1pt] (1,0) circle (1pt) coordinate (UE_A) node[below left] {A};
  \draw[fill=green,yshift=1pt] (.5,-0.5) circle (1pt) coordinate (UE_B) node[below left] {B};

  % Draw lines
  \draw (AP3) -- node[pos=0.6] {\SI{7.07}{m}} (UE_A);
  \draw (AP3) -- node[pos=0.6] {\SI{10}{m}} (UE_B);
\end{tikzpicture}

\end{document}

%%% Local Variables:
%%% mode: tikz
%%% TeX-PDF-mode: t
%%% TeX-master: t
%%% End:

