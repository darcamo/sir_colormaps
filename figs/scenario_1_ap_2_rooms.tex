\documentclass{standalone}

\usepackage[utf8]{inputenc} % Use this if the file is encoded with utf-8
\usepackage{lmodern}  % Very good to use with the fontenc package to generate good PDFs
\usepackage[T1]{fontenc}  % Important. See http://tex.stackexchange.com/questions/664/why-should-i-use-usepackaget1fontenc
\usepackage{amsmath,amssymb} % Part of AMS-LaTeX
\usepackage{siunitx}
% One of the good things of the amsmath package is the math enviroments matrix, pmatrix, bmatrix, Bmatrix, vmatrix and Vmatrix
\usepackage{tikz}
\usetikzlibrary{positioning}
\usetikzlibrary{shadows}
\usetikzlibrary{backgrounds}
\usetikzlibrary{shapes}
\usetikzlibrary{shapes.multipart}
\usetikzlibrary{matrix}
\usetikzlibrary{intersections}
\usetikzlibrary{fit}
%\usetikzlibrary{decorations.pathmorphing}
\usetikzlibrary{decorations.pathreplacing}
\usetikzlibrary{decorations.markings}
\usetikzlibrary{calc}

% My custom package with my math definitions
% Located at /home/darlan/Dropbox/Arquivos de Instalação/Latex_Packages/MathDefinitions.sty
\usepackage{MathDefinitions}

\begin{document}


\tikzset{AP/.style={draw, circle, fill=red, minimum size=8}}

\begin{tikzpicture}[scale=<<scale>>,transform shape]
  % We explicitly set the bounding box here so that the text nodes outside
  % the grid do not extend the bounding box.
  % \path[use as bounding box] (-6, -6) rectangle (6, 6);

  \draw[fill=gray] (-6, -6) grid (6, 6);

  \foreach \i in {-5.5, -3.5, ..., 5.5}
  {
    \foreach \j in {-5.5, -3.5, ..., 5.5} 
    {
      \draw ($(\i, \j) + (0cm,0)$) node[AP] {};
    } 
    \foreach \j in {-4.5, -2.5, ..., 5.5} 
    {
      \draw ($(\i, \j) + (1cm,0)$) node[AP] {};
    } 
  }

  \draw[yshift=-2mm] (-6, -6) -- node[below,scale=1.0] {\SI{10}{m}} (-5, -6);
  \draw[xshift=-2mm] (-6, -6) -- node[left,scale=1.0] {\SI{10}{m}} (-6, -5);
\end{tikzpicture}

\end{document}

%%% Local Variables:
%%% mode: tikz
%%% TeX-PDF-mode: t
%%% TeX-master: t
%%% End:

