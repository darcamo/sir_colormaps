\documentclass{article}

\usepackage[utf8]{inputenc}
%\usepackage[nolist]{acronym}
\usepackage{siunitx}
\usepackage{standalone}

% xxxxxxxxxx Tikz package xxxxxxxxxxxxxxxxxxxxxxxxxxxxxxxxxxxxxxxxxxxxxxxxx
\usepackage{tikz} % Create graphics in Latex
\usetikzlibrary{positioning}
\usetikzlibrary{shadows}
\usetikzlibrary{backgrounds}
\usetikzlibrary{shapes}
\usetikzlibrary{shapes.multipart}
\usetikzlibrary{matrix}
\usetikzlibrary{intersections}
\usetikzlibrary{fit}
%\usetikzlibrary{decorations.pathmorphing}
\usetikzlibrary{decorations.pathreplacing}
\usetikzlibrary{decorations.markings}
\usetikzlibrary{calc}
\usetikzlibrary{patterns}

% \input{figs/basestation}
% xxxxxxxxxxxxxxxxxxxxxxxxxxxxxxxxxxxxxxxxxxxxxxxxxxxxxxxxxxxxxxxxxxxxxxxxx


\title{<<title>>}
\author{<<author>>}
\date{\today}

% \begin{acronym}[16-QAM]
%   \acro{AN}{Access Node}
%   \acro{TC}{Test Case}
%   \acro{SIR}{Signal-to-Interference Ratio}
% \end{acronym}


\begin{document}

\maketitle


Fiquei interessado nos teus resultados para incluir nas aulas dos cursos de
comunicações moveis, aquela analise de SIR do grid de cômodos quadrados.

\section{Descrição do Cenário}

O cenário é composto de um andar de uma construção contendo \num{144} salas
quadradas com \SI{10}{m} de lado. A densidade de AN será variada de 1
AN em cada sala até 1 AN a cada \num{9} salas.

\begin{figure}[!ht]
  \centering
  % tikz code will not work when converting to html or docx. Therefore, we
  % will run latex filter in dexy for the figure and include the
  % resulting file using jinja templates
  %\documentclass{standalone}

\usepackage[utf8]{inputenc} % Use this if the file is encoded with utf-8
\usepackage{lmodern}  % Very good to use with the fontenc package to generate good PDFs
\usepackage[T1]{fontenc}  % Important. See http://tex.stackexchange.com/questions/664/why-should-i-use-usepackaget1fontenc
\usepackage{amsmath,amssymb} % Part of AMS-LaTeX
% One of the good things of the amsmath package is the math enviroments matrix, pmatrix, bmatrix, Bmatrix, vmatrix and Vmatrix
\usepackage{siunitx}
\usepackage{tikz}
\usetikzlibrary{positioning}
\usetikzlibrary{shadows}
\usetikzlibrary{backgrounds}
\usetikzlibrary{shapes}
\usetikzlibrary{shapes.multipart}
\usetikzlibrary{matrix}
\usetikzlibrary{intersections}
\usetikzlibrary{fit}
%\usetikzlibrary{decorations.pathmorphing}
\usetikzlibrary{decorations.pathreplacing}
\usetikzlibrary{decorations.markings}

% My custom package with my math definitions
% Located at /home/darlan/Dropbox/Arquivos de Instalação/Latex_Packages/MathDefinitions.sty
\usepackage{MathDefinitions}

\begin{document}


\tikzset{AP/.style={draw, circle, fill=red, minimum size=8}}

\begin{tikzpicture}[scale=0.4,transform shape]
  % We explicitly set the bounding box here so that the text nodes outside
  % the grid do not extend the bounding box.
  % \path[use as bounding box] (-6, -6) rectangle (6, 6);

  \draw[fill=gray] (-6, -6) grid (6, 6);

  \foreach \i in {-5.5, -4.5, ..., 5.5} 
  {
    \foreach \j in {-5.5, -4.5, ..., 5.5} 
    {
      \draw (\i, \j) node[AP] {};
    } 
  }

  \draw[yshift=-2mm] (-6, -6) -- node[below,scale=1.0] {\SI{10}{m}} (-5, -6);
  \draw[xshift=-2mm] (-6, -6) -- node[left,scale=1.0] {\SI{10}{m}} (-6, -5);
\end{tikzpicture}

\end{document}

%%% Local Variables:
%%% mode: tikz
%%% TeX-PDF-mode: t
%%% TeX-master: t
%%% End:


  \includegraphics{figs/scenario_1_ap_1_room.pdf}
  \caption{1 AN in each room}\label{fig:1_an_each_room}
\end{figure}

\section{Análise teórica simples}

Calcular a SIR analiticamente, pelo menos nas bordas.

\section{Medir a SIR em pontos
específicos}

\begin{itemize}
\itemsep1pt\parskip0pt\parsep0pt
\item
  Medir a SIR em pontos específicos que caracterizariam certos cenários

  \begin{itemize}
  \itemsep1pt\parskip0pt\parsep0pt
  \item
    os 4 cantos na borda e depois você constrói quadrados virtuais de
    raio menor, dentro do comodo
  \item
    por exemplo a 50\% do raio e a 10\% do raio
  \item
    e calcula a SIR nos 4 cantos desses quadrados virtuais,
    caracterizando situações de borda, meio de célula e próximo do AP
  \item
    Ai você calcula a media e os percentis 10\%, 50\% e 90\%.
  \end{itemize}
\end{itemize}

\section{Reduzir efeitos de borda}

\begin{itemize}
\itemsep1pt\parskip0pt\parsep0pt
\item
  Acho que vale a pena também reduzir os efeitos de borda

  \begin{itemize}
  \itemsep1pt\parskip0pt\parsep0pt
  \item
    incluir os cômodos de borda apenas na geração de interferência nos
    demais cômodos internos, mas excluir a SIR dos cômodos de borda nas
    estatísticas globais (podem enviesar os resultados)
  \end{itemize}
\end{itemize}

\section{Variação com diferentes layouts de disposição dos
APs}

\begin{itemize}
\itemsep1pt\parskip0pt\parsep0pt
\item
  Outra variação que pode ser verificada eh se cabem diferentes layouts
  de disposição dos APs no grid e como isso influencia os resultados.
\end{itemize}

\end{document}

%%% Local Variables:
%%% mode: latex
%%% TeX-master: t
%%% End:
